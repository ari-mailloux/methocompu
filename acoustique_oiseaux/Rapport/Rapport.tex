\documentclass[9pt,twocolumn,twoside,]{pnas-new}

% Use the lineno option to display guide line numbers if required.
% Note that the use of elements such as single-column equations
% may affect the guide line number alignment.


\usepackage[T1]{fontenc}
\usepackage[utf8]{inputenc}

% tightlist command for lists without linebreak
\providecommand{\tightlist}{%
  \setlength{\itemsep}{0pt}\setlength{\parskip}{0pt}}


% Pandoc citation processing
\newlength{\cslhangindent}
\setlength{\cslhangindent}{1.5em}
\newlength{\csllabelwidth}
\setlength{\csllabelwidth}{3em}
\newlength{\cslentryspacingunit} % times entry-spacing
\setlength{\cslentryspacingunit}{\parskip}
% for Pandoc 2.8 to 2.10.1
\newenvironment{cslreferences}%
  {}%
  {\par}
% For Pandoc 2.11+
\newenvironment{CSLReferences}[2] % #1 hanging-ident, #2 entry spacing
 {% don't indent paragraphs
  \setlength{\parindent}{0pt}
  % turn on hanging indent if param 1 is 1
  \ifodd #1
  \let\oldpar\par
  \def\par{\hangindent=\cslhangindent\oldpar}
  \fi
  % set entry spacing
  \setlength{\parskip}{#2\cslentryspacingunit}
 }%
 {}
\usepackage{calc}
\newcommand{\CSLBlock}[1]{#1\hfill\break}
\newcommand{\CSLLeftMargin}[1]{\parbox[t]{\csllabelwidth}{#1}}
\newcommand{\CSLRightInline}[1]{\parbox[t]{\linewidth - \csllabelwidth}{#1}\break}
\newcommand{\CSLIndent}[1]{\hspace{\cslhangindent}#1}


\templatetype{pnasresearcharticle}  % Choose template

\title{Template for preparing your research report submission to PNAS
using RMarkdown}

\author[]{Florence Beaudry}
\author[]{Jasmine Boulanger}
\author[]{Ariane Mailloux}
\author[]{Clara Surprenant}

  \affil[1]{Université de Sherbrooke, 2500 Boulevard de l'Université,
Sherbrooke, Québec, Canada}


% Please give the surname of the lead author for the running footer
\leadauthor{Mailloux}


\authorcontributions{}



\correspondingauthor{\textsuperscript{} }

% Keywords are not mandatory, but authors are strongly encouraged to provide them. If provided, please include two to five keywords, separated by the pipe symbol, e.g:


\begin{abstract}
Lors d'inventaires acoustiques de faune, la détectabilité est un aspect
qui peut mettre en jeu la qualité des données. Comment recenser la
présence d'un animal si l'on n'est pas capable de savoir qu'il est
présent? Dans un contexte de changements climatiques, la composition des
espèces dans les écosystèmes est en constant changement, menaçant
plusieurs espèces, comme la paruline du Canada (Cardellina canadensis)
(1). La détectabilité à différentes heures de la journée est établie
pour cette espèce, basée sur des inventaires effectués entre 2016 et
2020 au Québec. La quantité d'observations de différents individus et la
quantité d'observations de différentes espèces sont également établies.
Ces données permettent d'identifier les moments dans la journée où il
serait préférable de conduire des inventaires acoustiques, soit
xxxxxxxxxxxxxxxxx. Les inventaires pendant ces heures optimisent la
prise de données et augmentent la probabilité de détectabilité. Ceci
sera primordial dans l'étude des changements de biodiversité et dans le
suivi des populations de paruline du Canada.
\end{abstract}

\dates{This manuscript was compiled on \today}
\doi{\url{www.pnas.org/cgi/doi/10.1073/pnas.XXXXXXXXXX}}

\begin{document}

% Optional adjustment to line up main text (after abstract) of first page with line numbers, when using both lineno and twocolumn options.
% You should only change this length when you've finalised the article contents.
\verticaladjustment{-2pt}



\maketitle
\thispagestyle{firststyle}
\ifthenelse{\boolean{shortarticle}}{\ifthenelse{\boolean{singlecolumn}}{\abscontentformatted}{\abscontent}}{}

% If your first paragraph (i.e. with the \dropcap) contains a list environment (quote, quotation, theorem, definition, enumerate, itemize...), the line after the list may have some extra indentation. If this is the case, add \parshape=0 to the end of the list environment.

\hypertarget{introduction}{%
\section*{Introduction}\label{introduction}}
\addcontentsline{toc}{section}{Introduction}

La détection de faune par inventaire acoustique est une méthode répandue
afin d'établir la présence ou l'absence d'espèces dans un milieu, en
particulier pour les espèces vocales comme les oiseaux (Wimmer et al.,
2013). Le moment de la journée est documenté comme étant un facteur
d'influence pour l'activité des oiseaux, modulant la probabilité de
détecter ceux-ci selon l'heure d'échantillonnage (Wimmer et al., 2013).
De plus, différentes espèces sont actives à différents moments
(Perreault, 2010). La détectabilité des oiseaux peut entrer en jeu
lorsque les données se veulent le plus représentatives possible du
milieu. En particulier, dans l'étude des espèces à statut, il est
primordial de mettre toutes les chances du côté de l'observateur afin de
s'assurer de ne pas manquer des occurences de ces espèces. La Paruline
du Canada (Cardellina canadensis) est une espèce qui a récemment été
identifiée comme étant une espèce préoccupante par le Comité sur la
situation des espèces en péril au Canada (COSEPAC, 2020). L'aire de
répartition de cet oiseau chanteur s'étend sur toutes les provinces et
territoires du pays, en plus de couvrir certaines zones des États-Unis,
en particulier le long des Appalaches (COSEPAC, 2020). Malgré le déclin
annuel de 2,9\% connu par cette espèce entre 1970 et 2012, son statut
s'est amélioré dans les dernières années, puisque le rapport du COSEPAC
de 2008 lui avait accordé le statut «\,menacé\,» (COSEPAC, 2008;
Roberto-Charron et al., 2020). Le suivi de la répartition de C.
canadensis continuera donc à être essentiel afin de documenter la
progression de cette espèce et de prendre des mesures en conséquent. La
pression environnementale principale qui affecte les parulines du Canada
identifiée par Ferrari (2016) est le changement climatique. Plusieurs
autres espèces sont également affectées par cette menace, menant à des
modifications des structures des métapopulations dans plusieurs biomes
(Lovejoy \& Hannah, 2019). Il sera donc pertinent d'étudier l'impact des
changements climatiques sur les espèces occupant les mêmes milieux que
les parulines du Canada.

De l'échantillonnage acoustique a été effectué sur plusieurs sites au
Québec entre 2016 et 2020, permettant de recenser les espèces d'oiseaux
présentes dans de nombreux différents milieux. Les données récoltées
serviront à répondre aux questions suivantes\,:

À quel moment de la journée est-ce que le plus grand nombre d'oiseaux a
été détecté?

À quel moment de la journée est-ce que la plus grande diversité
d'espèces a été détectée?

À quel moment de la journée est-ce que le plus grand nombre de parulines
du Canada a été détecté?

\hypertarget{description-de-la-muxe9thode-et-des-ruxe9sultats}{%
\section*{Description de la méthode et des
résultats}\label{description-de-la-muxe9thode-et-des-ruxe9sultats}}
\addcontentsline{toc}{section}{Description de la méthode et des
résultats}

La prise de données s'est basée sur plusieurs périodes d'observation
d'environ 15 minutes prises entre 2016 et 2020 au Québec. Les oiseaux
détectés ont été répertoriés et identifiés avec le plus de détails
possibles ; en revanche, certaines identifications se sont rendues à
l'ordre, la famille, ou le genre sans atteindre l'identification de
l'espèce. Le niveau d'identification taxonomique a donc été rapporté
pour chaque observation. Les autres données incluent la date, les heures
de début et de fin de la période d'observation, l'heure de
l'observation, la latitude, et le numéro du site. Les données ont
d'abord été validées et nettoyées. Pour ce faire, plusieurs
vérifications de formatage et d'uniformité des données ont été
réalisées. Suite à cette étape, des traitements de données ont été
réalisés pour répondre aux questions de recherche.

Afin de répondre à la première question, soit l'heure à laquelle le plus
grand nombre d'oiseaux sont détectés, toutes les observations ont été
prises en considération, peu importe le niveau taxonomique atteint lors
de l'identification. Un graphique a été créé pour représenter le nombre
d'observations à chaque heure (FIGURE 1!!!!!!). SPECIFIER SI IL MANQUE
DES HEURES DANS LA FIGURE

Afin de déterminer l'heure à laquelle la plus grande diversité d'oiseaux
a été détectée, les données ont été triées pour seulement garder les
observations qui se rendaient à l'espèce. La qualité des données était
ainsi bonifiée pour éviter de compter la même espèce deux fois ou de
manquer des espèces. Par exemple, deux observations de paruline du
Canada pourraient avoir été recensées à différents niveaux ; l'une
d'elle à la famille, indiquant Parulidae, et l'autre à l'espèce,
indiquant Cardellina canadensis. Ces deux lignes auraient été comptées
comme des espèces différentes. À l'inverse, plusieurs observations de la
famille Parulidae pourraient indiquer une paruline du Canada et une
paruline triste, mais ces espèces seraient comptées comme une seule
espèce. Une fois les données triées, un graphique a été créé afin de
représenter la richesse spécifique à chaque heure (figure 2). SPECIFIER
SI IL MANQUE DES HEURES DANS LA FIGURE

Pour répondre à la troisième question, soit l'heure à laquelle les
parulines du Canada sont le plus détectées, seulement les observations
ayant identifié C. canadensis jusqu'à l'espèce ont été prises en compte.
Plus d'une centaine d'observations de cette espèce ont été recensées. Un
graphique représentant toutes ces observations selon l'heure a été
produit (figure 3). Celui-ci a permis d'identifier les heures de plus
grande activité des parulines du Canada, soit 5:00, 6:00 et 7:00 du
matin.

\hypertarget{discussion}{%
\section*{Discussion}\label{discussion}}
\addcontentsline{toc}{section}{Discussion}

Les résultats ont permis d'établir un profil journalier de détectabilité
pour toutes les espèces détectées ainsi que pour C. canadensis. Il est
important de noter que la prise de données montrait seulement les
périodes d'échantillonnage pour lesquelles il y avait eu des
observations. Donc, même si les résultats montrent une détectabilité
nulle pour certaines heures, soit minuit, 2:00, 9:00, 10:00, 11:00,
15:00, 16:00 et 17:00, il est probable qu'aucun effort d'échantillonnage
n'ait été déployé à ces heures. Il serait impossible de différencier une
détectabilité nulle d'une absence de période d'échantillonnage puisque
les périodes d'échantillonnage n'étaient pas recensées si aucun oiseau
n'était détecté.

En revanche, autant pour les parulines du Canada que pour les autres
espèces d'oiseaux, les heures de plus grande activité des oiseaux
étaient entre XXXXXXXXXXXXXXXXXXXX, ce qui correspond à ce qui est connu
présentement sur l'activité des oiseaux. En effet, il est généralement
conseillé de débuter l'observation acoustique d'oiseaux une demi-heure
avant le lever du soleil et de continuer jusqu'à 3 heures et demie à 4
heures et demie après le lever du soleil pour profiter du maximum
d'activité (Morgan et al., 1983).

1-À quel moment de la journée est-ce que le plus grand nombre d'oiseaux
a été détecté?

2-À quel moment de la journée est-ce que la plus grande diversité
d'espèces a été détectée?

3-À quel moment de la journée est-ce que le plus grand nombre de
parulines du Canada a été détecté?

À faire

\hypertarget{ruxe9fuxe9rences}{%
\section*{Références}\label{ruxe9fuxe9rences}}
\addcontentsline{toc}{section}{Références}

\showmatmethods
\showacknow
\pnasbreak

\hypertarget{refs}{}
\begin{CSLReferences}{0}{0}
\leavevmode\vadjust pre{\hypertarget{ref-cosepac_evaluation_2020}{}}%
\CSLLeftMargin{1. }%
\CSLRightInline{COSEPAC (2020) \emph{Évaluation et {Rapport} de
situation du {COSEPAC} sur la {Paruline} du {Canada} ({Cardellina}
canadensis) au {Canada}.} (Comité sur la situation des espèces en péril
au Canada, Ottawa).}

\end{CSLReferences}



% Bibliography
% \bibliography{pnas-sample}

\end{document}
